%
% Documentation for the mixer - a tool for the management of web sites.
%
% Copyright (C) 2003 Max Neunhoeffer, 
%                    Lehrstuhl D fuer Mathematik, RWTH Aachen
% 
% This file is protected under the GNU General Public License
% (see the file "GPL.txt" in die main distribution directory for details).
%
% $Id$
%

\documentclass[german,11pt]{article}

\parindent0pt
\usepackage{babel}
\usepackage[compat2,
            a4paper,nohead]{geometry}
\usepackage{times}
\usepackage[latin1]{inputenc}
\usepackage{color}
\definecolor{MyBlue}{rgb}{0.01,0.05,0.5}
\definecolor{MyGreen}{rgb}{0.01,0.4,0.05}
\definecolor{MyRed}{rgb}{0.7,0.01,0.05}
\usepackage[colorlinks=true,%hypertex,
            linkcolor=MyBlue,urlcolor=MyRed,citecolor=MyGreen,
            pdftitle={mixer - maintaining web sites easily},
            pdfauthor={Max Neunh\string\�ffer},
            pdfsubject={mixer},
            pdfkeywords={content management mixer web site tree}]{hyperref}

\newcommand{\mixer}{\texttt{mixer}}
\newcommand{\MIXERROOT}{\texttt{MIXERROOT}}

\begin{document}
\title{{\mixer} --- maintaining web sites easily}
\author{Max Neunh�ffer \\ \texttt{max.neunhoeffer@math.rwth-aachen.de}}
\date{\today}
\maketitle

\tableofcontents

\section{Introduction and Basic Idea}

If you have a rough idea what the {\mixer} is and just want to see quickly
how it all works, have a look at the examples in section \ref{examples}
and then read section \ref{hurry}. The gory details are then covered in
sections \ref{notation} to \ref{autonav}.

The {\mixer} is little Max's version of a "`Content Management System"'.
The goal is to help the user to maintain a web site consisting of
a hierarchy of web pages in a consistent way. Repeating parts of
these pages should each be stored and maintained in a single place
and an easily understandable procedure should make the final pages
from the input the user enters.

For each page this procedure basically pastes together various files
from different places in a well-defined way and replaces certain
elements by other stuff. Things like navigation tools for the visitors
of the site are generated automatically from available data.

The main design principles are:
\begin{itemize}
\setlength{\parsep}{0mm}
 \item SIMPLICITY,
 \item well-definedness,
 \item good documentation,
 \item the use of standard web technology (XML, XHTML, style sheets).
\end{itemize}

\section{Learning by example}
\label{examples}

Assume you have a directory \verb!DIR! and the following list of files
and directories:

\begin{verbatim}
DIR/MIXERROOT
DIR/lib/addresses
DIR/lib/config
DIR/lib/default.html
DIR/index.mixer
\end{verbatim}

The {\MIXERROOT} file is empty, \verb!addresses! is a address database
in a certain format and \verb!config! contains some variable settings.
Assume \verb!default.html! looks like this:

{\small
\begin{verbatim}
<?xml version="1.0" encoding="ISO-8859-1"?>
<html xmlns="http://www.w3.org/1999/xhtml" xml:lang="en" lang="en">
<head>
  <meta http-equiv="Content-Type" content="text/html; charset=ISO-8859-1" />
  <link href="/lib/default.css" type="text/css" rel="StyleSheet" />
  <style type="text/css" media="screen">@import url(/lib/default.css);</style>
  <title><mixer var="title">Here will be the title</mixer></title>
</head>
<body>
<mixer part="main">Here will be the main part of the page</mixer>
</body>
</html>
\end{verbatim}}

and \verb!index.mixer! looks like this:

\begin{verbatim}
<?xml version="1.0" encoding="ISO-8859-1"?>

<mixer template="default.tmpl"> 
<mixertitle>My title</mixertitle>

<h1><mixer part="title"/></h1>

<p>Current semester: <mixer var="semester"/></p>
<p>Author: <mixer person="Max Neunhoeffer" data="name_link"/></p>
<p>Some weird display, possibly navigation:<mixer func="myfunc"/></p>

</mixer>
\end{verbatim}

Then the {\mixer} does the following for you: 

It takes the file 
\verb!index.mixer!, reads off the name \verb!default.tmpl! from the
\verb!template! attribute and starts with \verb!default.tmpl! in the
\verb!lib! directory and produces a file \verb!index.html! by
replacing things.

The content from the file \verb!index.mixer! (beginning with the
\verb!<h1>! is inserted instead of 

\hspace*{5mm}\verb!<mixer part="main"/>!

and the title "`My title"' is inserted instead of \verb!<mixer part="title"/>!
in the head of the document and in the \verb!<h1>! element.

The value of the variable \verb!semester! is substituted for
\verb!<mixer var="semester"/>! and from the address database
a link to the personal page of Max Neunh�ffer together with his
name is inserted instead of 

\hspace*{1cm}\verb!<mixer person="Max Neunhoeffer" data="name_link"/>!

It is even possible to put the output of a user defined function \verb!myfunc!
in the final page instead of 

\hspace*{5mm}\verb!<mixer func="myfunc"/>!

This makes it for example possible to automatically generate navigation
tools for a whole web site.

This document is the manual to the {\mixer}, a program that achieves.


\section{If you are in a hurry}
\label{hurry}

Here is a table of replacements, that are performed:

\bigskip
\begin{tabular}{p{2in}cp{3.5in}}
  \verb!<mixer part="main"/>!  & $\to$ &
  the content of the current \verb!.mixer! file. \\
  \verb!<mixer part="title"/>! & $\to$ &
  the title of the current page from the \verb!mixertitle! element. \\
  \verb!<mixer var="uvw"/>!    & $\to$ &
  the value of the variable \verb!uvw! in the configuration database. \\
  \verb!<mixer part="abc"/>!   & $\to$ &
  the content of the part \verb!abc! of the current page which is found
  in the file \verb!xyz.abc! in the same directory, if we are working
  on the file \verb!xyz.mixer! currently. \\
  \hspace*{-3mm}
  \begin{tabular}[t]{l}\verb!<mixer person="ID"!\\\verb! data="KEY"/>!
  \end{tabular} & $\to$ &
  the entry under the key \verb!KEY! of the person with id \verb!ID! in
  the address database \\
  \verb!<mixer func="hij"/>!   & $\to$ &
  the result of the user defined function \verb!hij! \\
  \verb!<mixer klm="xyz"/>!    & $\to$ &
  the result of the user defined function \verb!klm!. Here \verb!klm! must
  not be equal to one of the other predefined attributes. \\
  \verb!href!-attributes       & $\to$ &
  links not containing a colon and starting with a slash are interpreted
  as links relative to the root of the site and are replaced by links
  relative to the current file, variable names in links, which are enclosed
  in double curly brackets, are replaced \\
  \verb!<link href="opq.css"/>! & $\to$ &
  the value of the \verb!href! is replaced by the value of the (optional) 
  \verb!oldstyle! attribute of the \verb!mixer! element in the
  \verb!.mixer! file. \\
  \hspace*{-5mm}
  \begin{tabular}[t]{l}\verb!<style>@import!\\\verb! url(opq.css);</style>!
  \end{tabular} & $\to$ &
  the link is replaced by the value of the (optional) \verb!style! attribute
  of the \verb!mixer! element in the \verb!.mixer! file.
\end{tabular}


\section{Notation}
\label{notation}

The {\mixer} feels responsible for a full subtree of the file system.
This tree is called "`the web site"' and the absolute path name to
its top is called \MIXERROOT. Individual HTML-files within this
subtree are called "`web pages"'.

\section{Installation of the \mixer}


First note that the {\mixer} needs Python version 2.2 or newer and that
this must be the version of Python coming up if one calls \verb!python!.

Untar the archive \verb!mixer.tar.gz! (possibly your version contains
some version number). This will create a directory \verb!mixer! where
the code resides. The {\mixer} is written in Python, however it uses
an extension of the Python interpreter which is written in C and uses
the C programm \verb!rxp! as an XML parser. Therefore some compilation
is needed. This is done by doing \verb!make! in the {\mixer} directory.

The only further step for installation is to create a symbolic link from
any position which is in your \verb!PATH! (for example \verb!/usr/local/bin!)
to the file \verb!mixer.py! in the {\mixer} directory.


\section{First steps for a new web site}


To get things going, you need at least:

A directory with an empty file with the name \verb!MIXERROOT! and a
subdirectory \verb!lib!.

You need a file with the name \verb!addresses! in the \verb!lib! directory.
It is a valid XML file with only one element of type \verb!addresses! that
contains elements of type \verb!person! with arbitrary attributes.

You further need a file with the name \verb!config! in the \verb!lib! 
directory. It must be a valid Python module. You can define variables
in there in Python notation. All variable values must be strings.

The file \verb!funcs.py! in the \verb!lib! directory is optional and 
only used if you want to use user defined functions.

To see the {\mixer} doing something you need at least one template file
and a file with \verb!.mixer! at the end.


\section{Finding the \MIXERROOT}

The {\mixer} can be invoked from anywhere within the file system tree of the
web site. It determines the {\MIXERROOT} by looking upwards for an empty
file with the name \MIXERROOT. Here are the details:

After invocation {\mixer} first determines an absolute path to the
current working directory. This is done by looking at the environment
variable \verb!PWD!, which is set by most shells to an absolute path
to the current working directory. However, this can contain symbolic links
embedded in the path. So {\mixer} checks whether this directory is the
same as the directory that the C-library function \texttt{getcwd()}
returns. If both paths refer to the same physical directory, then
the value of \verb!PWD! is taken, otherwise the result of \texttt{getcwd()}.

The {\mixer} then goes up in the file tree by removing parts at the end of
the path, until it finds an empty file with the name \MIXERROOT. The
path pointing to the corresponding directory is the \MIXERROOT.

The reason for taking the value of \verb!PWD! is that symbolic links
within this path should be preserved. The comparison with the result of
\texttt{getcwd()} is done just to be save. Note that the file system subtree
below {\MIXERROOT} \textbf{must not} contain symbolic links to subdirectories.

\section{How does the {\mixer} put together web pages?}
\label{puttogether}

The {\mixer} first reads the configuration database in
\MIXERROOT\verb!/lib/config! and then the address database, which is in
the file \MIXERROOT\verb!/lib/addresses!. 

After that it interprets the
file \MIXERROOT\verb!/lib/funcs.py! (if it exists) as a Python script
to get hold of the user specified functions. Then it walks recursively
through the whole subtree below {\MIXERROOT} and does the following job
on all files ending in \verb!.mixer!:

Assume it works on a file \verb!xyz.mixer!, which has to be a valid
XML-file with top level element of type \verb!mixer!, having an
attribute \verb!template!. The value of this attribute has to be
one of the template documents (without any path). The referenced template
document also has to be a valid XML-file and has therefore a natural
tree structure. The {\mixer} starts with this tree and recursively
walks through it, replacing the subtree corresponding to the element
\verb!<mixer part="main"/>! by the tree defined by \verb!xyz.mixer!.
Directly after the replacement this tree is walked through and only
after that is done the walk through the first tree continues.

During these walks various replacements of subtrees are done, which are
explained in the following sections. As described in section \ref{partsubs}
for "`part substitution"' other files with names like \verb!xyz.ABC!
are used, where \verb!ABC! is replaced by the name of the requested
part.

There are possibilities to insert the value of a variable defined
in the configuration database, to insert values from the address
database, to insert other parts or to insert the result of a user
defined function.

After all these replacements, the resulting tree is written out as
a valid XHTML document (provided the input was correct) to the file
\verb!xyz.html!.

The {\mixer} usually only works on those files \verb!xyz.mixer! which
have a newer modification date than the corresponding file
\verb!xyz.html!, unless the command line option \verb!-f! for "`force"'
is used. In the latter case or if any of the template documents is 
modified more recently, the complete site is rebuilt.

\section{Template documents}
\label{templatedocs}

Template documents play the role of a common frame for many web pages.
They are typically ending in \verb!.tmpl! and can reside anywhere
in the file tree below the {\MIXERROOT}.
Their format is easily described: A template document
must be a valid XHTML document, apart from certain additional elements
of type \verb!mixer!, which are replaced during the work of the {\mixer}
as described below. However, to make sense, a template document should adhere
to the following conventions or rules:

A template document should \textbf{not} contain a \verb!DOCTYPE! declaration.
This is inserted by the {\mixer} before writing out the HTML-files. The 
reason for this is that with the \verb!mixer! elements it \textbf{is not}
a valid XHTML document and with the \verb!DOCTYPE! declaration the
XML parser within the {\mixer} would complain.

A template document must have an element of type \verb!title!, however
its content does not matter. This is a rule from the XHTML standard.

A template document should contain an old style and a new style declaration
of a style sheet, as follows (in this order!):

{\small
\begin{verbatim}
<link href="/lib/default.css" type="text/css" rel="StyleSheet" />
<style type="text/css" media="screen">@import url(/lib/default.css);</style>
\end{verbatim}}

Note that the seemingly absolute links will be replaced by the {\mixer}
as described in section \ref{links}. Modern browsers will read both
declarations and the values in the second will take precedence over
those in the first. Netscape version 4.xx (and probably earlier) will
not read the second. This is a way to deal with the inherently broken
implementation of cascading style sheets in netscape version 4.xx. If
need be, one can easily specify a different stylesheet for netscape
4.xx.

A template document should somewhere in the body contain an element
\verb!<mixer part="main"/>!. Otherwise all pages referring to this
template document will be equal and the main part is not included at all.

The template document used for a certain \verb!.mixer! file is determined
by the \verb!template! attribute of the top level \verb!mixer! element
in the \verb!.mixer! file. The attribute value should not contain a path
but only a filename. This file is searched from the position of the
\verb!.mixer! file upwards until the {\MIXERROOT} and the first file
found with this name is taken. If no file is found the {\mixer} also
searches in {\MIXERROOT}\verb!/lib!.


\section{What replacements does the {\mixer} perform?}

During its walk "`through the tree"' (see section \ref{puttogether}), the
{\mixer} replaces all elements of type \verb!mixer!. Actually it replaces
the whole subtree below such an element. Therefore with respect to the
final result it does not matter, whether you write

\hspace*{1cm} \verb!<mixer part="main"/>!

or

\hspace*{1cm} \verb!<mixer part="main">Here will be the main part.</mixer>!

However since most browsers will ignore the \verb!mixer! tags that they
do not know and display only the text "`Here will be the main part."',
the second variant has an advantage: You can look at the template document
with a web browser and will see the place, where later the actual
content will be placed. This hint of course holds for all replacements
described in the following sections.

\subsection{Variable substitution}

Any element of type \verb!mixer! having an attribute \verb!var! with
value \verb!x! is replaced by the value of the variable \verb!x! in 
the configuration database. 

Example:

\begin{verbatim}
<mixer var="a"/>
\end{verbatim}

The configuration database is just
the file \MIXERROOT\verb!/lib/config!, which must be a Python script
that only sets some variables to string values. Note that variable
assignment is done with \verb!=! in Python and that string values have
to be enclosed in either single or double quotes. If you want to
define string constants that contain more than one line, you have to
use either triple quotes. Some examples:

\begin{verbatim}
a = 'Max'
b = "Till"
c = '''Hi
there!'''
d = """This even contains single quotes ' and
multiple lines!"""
\end{verbatim}

There is a special variable with name \verb!today! which is replaced
by the result of the C-library function \verb!ctime!. Note that of 
course this will give you the date of the time when the \verb!mixer!
ran last!

\subsection{Part substitution}
\label{partsubs}

Assume the mixer currently works on a file \verb!x.mixer! in some
directory \verb!dir!.

Any element of type \verb!mixer! having an attribute \verb!part! is replaced
by a full tree from another file. If the value of the \verb!part! attribute
is \verb!main!, the tree from \verb!x.mixer! is used.
If the value is \verb!title!, a special case occurs, which is described
in the following section. For all other values \verb!y!, the content of 
the file \verb!x.y! in the directory \verb!dir! is taken.

Example:

\begin{verbatim}
<mixer part="main"/>
\end{verbatim}

Note that in all cases the inserted subtree is traversed recursively
before the rest of the first tree.


\subsection{The title of a page}

There is a special case of part substitution, which was already mentioned 
in the last section, namely if the \verb!part! attribute has the value
\verb!title!. Very often the title of a page will occur not only in
the \verb!title! element of the head of the page, but also in some
heading in the main page. Therefore the following mechanism was invented.

When the main part is read from a \verb!.mixer! file, the top level of
the corresponding tree is scanned for an element of type \verb!title!.
If it is found, it is removed in this tree and stored separately.
Its content is then placed not only into the \verb!title! element
in the head of the page but also inserted at all places, where an
element like \verb!<mixer part="title"/>! appears. Note that
recursive replacement takes place within this title subtree, such
that for example variables can be used inside.


\subsection{Style sheets}

To configure the usage of different stylesheets conveniently some extra
magic has been implemented in the \mixer. If the top level element
of a \verb!.mixer! file has an attribute \verb!style!, its value is
placed into the new style stylesheet declaration of type \verb!style!
in the head of the document. Likewise, the value of an attribute
\verb!oldstyle! is placed into the old style stylesheet declaration of
type \verb!link! (see section \ref{templatedocs}).

\subsection{Links}
\label{links}

For all links in the trees a convenience procedure is performed. All
values of \verb!href! attributes are considered to be links. In addition
the value in the \verb!@import url(...);! statement in the
\verb!style! element in the head is also considered a link.

First a special form of variable substitution is performed: If the link
contains constructs in double braces, the text between the braces is
taken as the name of a variable from the configuration database 
and the brace expression is substituted by the value of the variable.

Afterwards the link is changed into a relative link in the following way:

If a link contains a colon, it remains untouched, because it is considered
to be an external link, probably containing \verb!http://! or a similar
thing.

If a link does not start with a slash, it remains untouched, because it
is considered to be an internal, relative link "`as is"'.

If a link does not contain a colon and starts with a slash, it is
considered to be an internal link, which should be relative to the
current position in the end. It is interpreted as a path relative to
the {\MIXERROOT} and is changed to a relative link from the current
position, usually by prepending a certain number of repetitions of
"`\verb!../!"' for "`go one up"'. 

This last feature is convenient, because one can always refer to other
documents via their "`absolute"' path with respect to the {\MIXERROOT}
and gets automatically relative links in the end.


\subsection{User defined functions}

If the {\mixer} finds a construct like \verb!<mixer func="xyz"/>!, it
calls the function \verb!xyz! which must be defined in the file
\MIXERROOT\verb!/lib/funcs.py! which is interpreted at startup time.
This function gets five arguments. The first is an absolute path name
to the \MIXERROOT, which ends in a slash. The second is a path name
leading from the {\MIXERROOT} to the directory in which the current
\verb!.mixer! file resides. The third argument is the name of the current
\verb!.mixer! document without the extension \verb!.mixer!. The fourth
argument is the complete tree of the main part in the memory representation
described below. The fifth argument is the current subtree in the same form.

The function
must return something which can be worked on by the "`walking"' routines
in the {\mixer}. 

The return value can be a string. In this case it is just put at the place
of the \verb!mixer! element. If it is a list, all entries in the list
are inserted instead of the \verb!mixer! element (and therefore are again
subject to the same requirements).

The return value also can be a full subtree in the memory representation of
XML-trees within the {\mixer}. This representation works as follows:
An element is represented by an object in the class \verb!xmltree! which 
is defined in the module \verb!maxml!. This class has four data entries:
\verb!type! is a string containing the name of the element. \verb!attr!
is either \verb!None! or a Python dictionary containing the attributes
of the object. \verb!subs! is either \verb!None! for an empty element
or a list of children. Those children can either be strings or subelements
in the form of objects in the class \verb!xmltree! recursively.

Note the difference between the value \verb!None!
and an empty list: The first indicates that the element had only
one opening and closing tag like \verb!<element/>!, the second
indicates, that it has opening and closing tags, but no children.

The last entry is called \verb!meta! and contains meta data like source
positions and is not used for further processing.

Note that \verb!mixer! elements with function calls need not be
empty. User defined functions have access to the subtree via the
fifth argument.

\subsection{Short form of call to user defined functions}

As convenience for web site authors there is a short form to call
user defined functions:

\begin{verbatim}
<mixer myfunc="Argg!"/>
\end{verbatim}

If in an \verb!mixer! element has exactly one attribute key and this is not 
among the predefined names described in this section, the \verb!mixer! assumes
that this attribute key (in the example above "`\verb!myfunc!"') is the name
of a user defined function, which is called as in the long form described
in the last subsection.

\section{The address database and its applications}

The address database in the file \MIXERROOT\verb!/lib/addresses! is of
course again an XML-document. It must consist of a single element
of type \verb!addresses!, which contains a sequence of elements
of type \verb!person!. Those latter elements all have to have an
\verb!id! attribute for identification and may have any other 
attributes. Standard attributes in use are:

\texttt{name, formalname, title, department, university, building, street, 
city,}

\texttt{count, zipcode, country, www, email, fax, phone}

If one uses the attribute \verb!sameaddressas! with the
\verb!id! of another person as value, the address information of the
other person is copied unless specified otherwise.

The information in the address database can be inserted into web pages
by using elements like 

  \hspace*{1cm}\verb!<mixer person="xyz" data="abc"/>!

where
\verb!xyz! must be the \verb!id! of a person in the database and
\verb!abc! must be an attribute specified for that person.

Note that some additional attributes are generated automatically, if the
necessary data is available:

\hspace*{1cm}
\begin{tabular}{lp{4.3in}}
  \verb!name_link! &
    for a link to the person's web page which has his name as text of
    the link.\\
  \verb!name_link_email! &
    for the above link together with a clickable email address.\\
  \verb!email_link! &
    for a "`mailto:"' link with the email as visible text.\\
  \verb!title_name! &
    for the full name with title. \\
  \verb!address! &
    for the full address consisting of the following entries, if given:\\
    & \verb!department!, \verb!university!, \verb!building!, \verb!street!, 
      \verb!county!, \\
    & \verb!city!, \verb!zipcode!, \verb!country!. \\
  \verb!contact! &
    the address as in \verb!address! plus an email address, if given. \\
  \verb!name_city! &
    the name and the city, if both are given, otherwise only the name. \\
  \verb!name_link_city! &
    the name linked to the person's web page, and the city, if both are 
    given, otherwise only the name.
\end{tabular}


\section{Automatic generation of site navigation}
\label{autonav}

The current distribution of the {\mixer} contains a sample implementation
of navigation generating functions in \MIXERROOT\verb!/lib/funcs.py!.
In this approach one assumes that the whole site has a "`spanning tree"',
which is mirrored in the file system tree. Every node in this tree
corresponds to web page. Every internal node (i.e. not a leaf) corresponds
to a directory in the file system and every leaf corresponds to a page
within the directory of the parent node.

This tree is configured by putting a file with the name \verb!tree! in
each directory within the web site. Such a \verb!tree! file must be
a valid XML-file with one top element of type \verb!node!, that has
an attribute \verb!file! which contains the name of the web page
corresponding to the node of the tree belonging to this directory.
The top element contains a sequence of elements of type \verb!entry!
which declare the children in the tree. Note that the order of the
entries matters for navigation purposes. Each \verb!entry! node
has an attribute \verb!file!, the value of which refers to either
a subdirectory (if the child is not a leaf) or to a file in the
current directory (if the child is a leaf). The content of the
\verb!entry! element is the text that should appear in the navigation
tools.

These are the two functions which generate such navigation tools:

\hspace*{3mm}
\begin{tabular}{lp{3.5in}}
  \verb!<mixer func="maketop"/>! &
  for a function producing a top line with links to the children of
  the root node \\
  \verb!<mixer func="makeleft"/>! &
  for a function producing a vertical bar visualizing the current position
  in the tree by displaying the current path and the nodes in the
  vicinity. The idea is that of a bar appearing to the left of the
  main content by using a table.
\end{tabular}
\end{document}
